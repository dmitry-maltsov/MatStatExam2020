\documentclass[a4]{article}
\pagestyle{myheadings}

%%%%%%%%%%%%%%%%%%%
% Packages/Macros %
%%%%%%%%%%%%%%%%%%%
\usepackage{mathrsfs}


\usepackage{fancyhdr}
\pagestyle{fancy}
\lhead{}
\chead{}
\rhead{}
\lfoot{}
\cfoot{} 
\rfoot{\normalsize\thepage}
\renewcommand{\headrulewidth}{0pt}
\renewcommand{\footrulewidth}{0pt}
\newcommand{\RomanNumeralCaps}[1]
{\MakeUppercase{\romannumeral #1}}

\usepackage{amssymb,latexsym}  % Standard packages
\usepackage[utf8]{inputenc}
\usepackage[russian]{babel}
\usepackage{MnSymbol}
\usepackage{amsmath,amsthm}
\usepackage{indentfirst}
\usepackage{graphicx}%,vmargin}
\usepackage{graphicx}
\graphicspath{{pictures/}} 
\usepackage{verbatim}
\usepackage{color}









\DeclareGraphicsExtensions{.pdf,.png,.jpg}% -- настройка картинок

\usepackage{epigraph} %%% to make inspirational quotes.
\usepackage[all]{xy} %for XyPic'a
\usepackage{color} 
\usepackage{amscd} %для коммутативных диграмм


\newtheorem{Lemma}{Лемма}[section]
\newtheorem{Proposition}{Предложение}[section]
\newtheorem{Theorem}{Теорема}[section]
\newtheorem{Corollary}{Следствие}[section]
\newtheorem{Remark}{Замечание}[section]
\newtheorem{Definition}{Определение}[section]
\newtheorem{Designations}{Обозначение}[section]




%%%%%%%%%%%%%%%%%%%%%%%% 
%Сношение с оглавлением% 
%%%%%%%%%%%%%%%%%%%%%%%% 
\usepackage{tocloft} 
\renewcommand{\cftdotsep}{2} %частота точек
\renewcommand\cftsecleader{\cftdotfill{\cftdotsep}}
\renewcommand{\cfttoctitlefont}{\hspace{0.38\textwidth} \LARGE\bfseries} 
\renewcommand{\cftsecaftersnum}{.}
\renewcommand{\cftsubsecaftersnum}{.}
\renewcommand{\cftbeforetoctitleskip}{-1em} 
\renewcommand{\cftaftertoctitle}{\mbox{}\hfill \\ \mbox{}\hfill{\footnotesize Стр.}\vspace{-0.5em}} 
\renewcommand{\cftsubsecfont}{\hspace{1pt}} 
\renewcommand{\cftparskip}{3mm} %определяет величину отступа в оглавлении
\setcounter{tocdepth}{5} 




\addtolength{\textwidth}{0.7in}
\textheight=630pt
\addtolength{\evensidemargin}{-0.4in}
\addtolength{\oddsidemargin}{-0.4in}
\addtolength{\topmargin}{-0.4in}

\newcommand{\empline}{\mbox{}\newline} 
\newcommand{\likechapterheading}[1]{ 
	\begin{center} 
		\textbf{\MakeUppercase{#1}} 
	\end{center} 
	\empline} 

\makeatletter 
\renewcommand{\@dotsep}{2} 
\newcommand{\l@likechapter}[2]{{\bfseries\@dottedtocline{0}{0pt}{0pt}{#1}{#2}}} 
\makeatother 
\newcommand{\likechapter}[1]{ 
	\likechapterheading{#1} 
	\addcontentsline{toc}{likechapter}{\MakeUppercase{#1}}} 





\usepackage{xcolor}
\usepackage{hyperref}
\definecolor{linkcolor}{HTML}{000000} % цвет ссылок
\definecolor{urlcolor}{HTML}{AA1622} % цвет гиперссылок

\hypersetup{pdfstartview=FitH,  linkcolor=linkcolor,urlcolor=urlcolor, colorlinks=true}



\def \newstr {\medskip \par \noindent} 


\begin{document}
\tableofcontents
\newpage

	\section{1}
	\subsection*{Формулировка}
	Формы представления данных и цели анализа данных.
	\subsection*{Ответ}
	По форме представления данных можно выделить выборк, функции, семантические данные(тексты, изображения). В качестве основных целей анализа данных можно отметить анализ глобальных событий и явлений, компактное представление данных, прогнозирование.
	
	\section{2}
	\subsection*{Формулировка}
	Характеристики положения данных.
	\subsection*{Ответ}
	1)Выборочное среднее, 2)med x = x ? x = x[k+1] || x = x[k], в случае нечётности и чётности соответственно(выборочная медиана), 3)$Z_R$ = полусумма экстремальных значений, 4)Полусумма выборочных квартилей $Z_Q = \frac{z_{1/4} + z_{3/4}}{2}$, 5)$Z_{tr}$ = $\frac{1}{n - 2r} \sum_{i = r+1}^{n-r} x_{(i)}$ - усечённое среднееб 6)Среднее геометрическое $Z_G$ = ${x_1 *...* x_n}^{1/n}$, 7)Среднее гармоническое $Z_M = \frac{1}{1/n * \sum_{i = 1}^{n}1/z_i}$, 8)Среднее по Колмогорову $z_k = g^{-1} (\frac{1}{n} \sum_{i = 1}^n g(x_i))$.\\
	Вариационный ряд данных - неубывающая выборка.
	
	\section{3}
	\subsection*{Формулировка}
	Характеристики рассеивания данных.
	\subsection*{Ответ}
	1)Среднеквадратичное отклонение и выборочная дисперсия, 2)среднее абсолютное отклонение от медианы d = 1/n $\sum_{i =1}^{n} |x_i - med x|$, 3)R = $X_n - X_1$ - размах выборки, 4)интервальная широта, 5)Медианное абсолютное отклонение
	
	\section{4}
	\subsection*{Формулировка}
	Оптимизационный подход к построению х-к положения и рассеивания данных.
	\subsection*{Ответ}
	
	
	
	\section{5}
	\subsection*{Формулировка}
	Характеристики взаимосвязи данных.
	\subsection*{Ответ}
	Коэффициент корреляции Пирсона(линейная зависимость), ранговый коэффициент Спирмена - мера монотонной зависимости(не линейной), квадратный (знаковый) коэффициент корреляции.
	
	\section{6}
	\subsection*{Формулировка}
	Характеристики экстремальных значений данных.
	\subsection*{Ответ}
	Нужно для выявления выбросов.\\
	1)|$x_i - med x$| > K * MAD x -> IQR\\
	2)$x_i$ - выброс, если > $max(x_1, LQ - 3/2 IQR) || < min(x_n, UQ + 3/2 IQR)$.
	
	
	\section{7}
	\subsection*{Формулировка}
	Графическое представление данных –боксплот Тьюки.
	\subsection*{Ответ}
	IQR = UQ - LQ. $x_i$ - выброс, если > $max(x_1, LQ - 3/2 IQR) || < min(x_n, UQ + 3/2 IQR)$.
	
	\section{8}
	\subsection*{Формулировка}
	Характеристики распределений данных: «ядерные» оценки плотности.
	\subsection*{Ответ}
	Эмпирическая функция распределения, дельта функция.\\
	Ядро - функция K(u), если : 1. K(u) $\geq$ , 2. K(-u) = K(u), 3. $\int_{-\infty}^{\infty} K(u) du = 1$.\\
	Если функция обладает первым свойством, то результатом ядерной оценки плотности действительно будет плотность вероятности. Третье свойство гарантирует, что среднее значение распределения равно среднему использованной выборки.\\
	Нужно для оценки плотности распределения. f(x) = $lim_{n -> \infty, h -> 0} 1/n * h \sum_{i = 1}^n K(\frac{x-x_i}{h})$
	
	
	\section{9}
	\subsection*{Формулировка}
	Что такое точечная оценка?
	\subsection*{Ответ}
	Оценка параметра — соответствующая числовая характеристика, рассчитанная по выборке. Оценки параметров генеральной совокупности делятся на два класса: точечные и интервальные.\\
	Когда оценка определяется одним числом, она называется точечной оценкой. Точечная оценка, как функция от выборки, является случайной величиной и меняется от выборки к выборке при повторном эксперименте.\\
	К точечным оценкам предъявляют требования, которым они должны удовлетворять, чтобы хоть в каком-то смысле быть «доброкачественными». Это несмещённость, эффективность и состоятельность.\\
	Несмещённость - если мат ожидание оценки равно оцениваемому параметру генеральной совокупности. Эффективность - если обладает минимальной дисперсией среди всех несмещенных точечных оценок. Состоятельность - если при увеличении выборки стремится по вероятности к параметру генеральной совокупности. Генеральная совокупность - параметр от которого зависит выборка.\\
	
	\section{10}
	\subsection*{Формулировка}
	Что такое статистика?
	\subsection*{Ответ}
	Статистика — отрасль знаний, наука, в которой излагаются общие вопросы сбора, измерения, мониторинга, анализа массовых статистических (количественных или качественных) данных и их сравнение; изучение количественной стороны массовых общественных явлений в числовой форме.
	
	\section{11}
	\subsection*{Формулировка}
	Какая оценка называется состоятельной, несмещенной, эффективной, робастной?
	\subsection*{Ответ}
	Несмещённость - если мат ожидание оценки равно оцениваемому параметру генеральной совокупности. Эффективность - если обладает минимальной дисперсией среди всех несмещенных точечных оценок. Состоятельность - если при увеличении выборки стремится по вероятности к параметру генеральной совокупности. Генеральная совокупность - параметр от которого зависит выборка. Робастная - устойчивая к выбросам\\
	
	\section{12}
	\subsection*{Формулировка}
	Какая из двух оценок считается более эффективной?
	\subsection*{Ответ}
	ДЛЯ НЕПРЕРЫВНЫХ СЛУЧАЙНЫХ ВЕЛИЧИН ВЕРОЯТНОСТЬ ТОГО, ЧТО ТОЧЕЧНАЯ ОЦЕНКА (ширина доверительного интервала равна 0) СОВПАДЕТ С ЛЮБЫМ ЗАДАННЫМ ЗНАЧЕНИЕМ ИЛИ ОЦЕНИВАЕМЫМ ПАРАМЕТРОМ РАВНА 0.\\
	Таким образом, точечная оценка имеет смысл лишь тогда, когда приведена характеристика рассеяния этой оценки (дисперсия). В противном случае она может служить лишь в качестве исходных данных для построения интервальной оценки.\\
	Интервальная оценка лучше.
	
	\section{13}
	\subsection*{Формулировка}
	Что такое эффективность, относительная эффективность, асимптотическая эффективность оценки?
	\subsection*{Ответ}
	
	
	\section{15}
	\subsection*{Формулировка}
	Приведите примеры состоятельных оценок м.о. нормального распределения.
	\subsection*{Ответ}
	
	\section{16}
	\subsection*{Формулировка}
	Приведите примеры состоятельных оценок м.о. распределения Лапласа.
	\subsection*{Ответ}
	
	\section{17}
	\subsection*{Формулировка}
	Приведите примеры состоятельных оценок м.о. равномерного распределения.
	\subsection*{Ответ}
	
	\section{18}
	\subsection*{Формулировка}
	Приведите примеры состоятельных оценок центра симметрии распределения Коши.
	\subsection*{Ответ}
	
	\section{19}
	\subsection*{Формулировка}
	
	\subsection*{Ответ}
\end{document}